\section{Общие сведения}
В настоящее время предприятие \enquote{РКЦ \enquote{Прогресс}} входит в государственную корпорацию \enquote{Роскосмос}.\\
В составе \enquote{Роскосмоса} \enquote{Прогресс} выполняет задачи по разработке и производству ракет-носителей, блоков выведения, а также некоторой авиационной и народно-хозяйственной продукции.\\
В состав ракетно-космического центра входят:
\begin{itemize}
    \item Байконурский филиал (Байконур, Казахстан)
    \item Краснознаменский филиал (Краснознаменск, Московская область)
    \item Научно-производственное предприятие \enquote{Оптико-электронные комплексы и системы} (Зеленоград, Москва)
    \item ОКБ \enquote{Спектр} (Рязань)
    \item Плесецкое представительство (Мирный, Архангельская область)
    \item Московское представительство (Москва)
\end{itemize}
История становления и развития организации представлена далее в таблице 1.\\\\
\begin{tabular}{|c|c|m{4.5cm}|m{8cm}|}
    \hline
    \textnumero & Этап & Наименование & Задачи \\
    \hline
    1. & 1894 & Фабрика \enquote{Дукс} & Немцем Ю. Меллером основана фабрика по производству велосипедов.\\
    \hline
    2. & 1900 & АО \enquote{Дукс} & Фирма преобразована в акционерное общество, помимо велосипедов запускается производство автомобилей, мотоциклов, дрезин и т. д.\\
    \hline
    3. & 1918--1919 & Государственный авиационный завод \textnumero 1 & Советом Народных Комиссаров принят Декрет о национализации предприятия.\\
    \hline
    4. & 1923 & ГАЗ \textnumero 1 & Начало серийного производства самолётов. Сначала лёгкого бомбардировщика Р-1, затем истребителей И-1, И-3, И-15. \\
    \hline
    5. & 1939 & ГАЗ \textnumero 1 & Пуск первой в мире двухступенчатой ракеты, изготовленной на ГАЗ \textnumero 1.\\
    \hline
    6. & 1940 & ГАЗ \textnumero 1 & Начало серийного производства истребителей МиГ-3, а затем и штурмовиков ИЛ-2 и ИЛ-10.\\
    \hline
    7. & 1958--1959 & ГАЗ \textnumero 1 & Правительством принято Постановление об организации серийного выпуска межконтинентальных баллистических ракет на базе ГАЗ \textnumero 1.\\
    \hline
    8. & 1961 & \enquote{Завод \enquote{Прогресс}} & Производство баллистических ракет и ракет-носителей.\\
    \hline
    9. & 1996 & Государственный научно-производственный ракетно-космический центр \enquote{ЦСКБ-Прогресс} & Объединение ЦСКБ и завода \enquote{Прогресс} в одну структуру. Задачами остаются проектирование и производство ракет-носителей, космических аппаратов, предназначенных для дистанционного зондирования Земли и фоторазведки, а также научные КА.\\
    \hline
    10. & 2014 & АО \enquote{РКЦ \enquote{Прогресс}} & Указом Президента РФ реорганизован в акционерное общество. Сначала 100\% акций принадлежали государству, затем были переданы корпорации \enquote{Роскосмос}\\
    \hline

\end{tabular}

\pagebreak

Сейчас предприятие производит следующую продукцию:

\pagebreak