\renewcommand{\baselinestretch}{1.5}
\addcontentsline{toc}{section}{Введение}
\section*{Введение}

Организация \enquote{РКЦ \enquote{Прогресс}} ведёт свою деятельность в космической отрасли.\\
Как утверждается на официальном сайте, это ведущее предприятие и один из лидеров мировой космической отрасли по разработке, производству и эксплуатации ракет-носителей среднего класса. Ракеты-носители, созданные на этом предприятии, используются для доставки полезной нагрузки на МКС. Сказано также, что космические аппараты, выпускаемые предприятием имеют высокий уровень надёжности, благодаря чему заслужили признание на международном уровне.\\
Космическая промышленность России является правопреемницей советской государственной космической индустрии и достаточно хорошо развита.\\
Целями космической программы РФ называется формирование и поддержание необходимого состава орбитальной группировки КА (спутников), обеспечивающих предоставление услуг в социально-экономической сфере, сфере науки и международного сотрудничества, в том числе в целях защиты населения и территорий от чрезвычайных ситуаций природного и техногенного характера. А также создание средств выведения и технических средств, создание научно-технического задела для перспективных космических комплексов и систем.\\
Однако, у космической отрасли есть и проблемы. В первую очередь, это сильная нагрузка на государственный бюджет. Эту проблему, как считают эксперты, надо решать привлечением частного капитала, а для этого необходимо повысить престиж космической отрасли, как это было в 1960-х годах.\\
Также среди проблем можно выделить некоторый \enquote{застой} отрасли: не создаются новые предприятия и направления развития, а старые предприятия нуждаются в реструктуризации, техническом и промышленном переоснащении.\\
Сама по себе отрасль имеет большое значение для народного хозяйства: при помощи спутников и зондов ведут метеорологические наблюдения для предсказания погоды и природных явлений, также спутники обеспечивают навигацию (GPS и ГЛОНАСС), используются для установления связи (спутниковые телефоны, а с недавнего времени и спутниковый интернет Starlink от SpaceX).\\
Предприятие {РКЦ \enquote{Прогресс}} решает такие задачи отрасли, как проектирование и эксплуатация ракет-носителей, при помощи которых выводятся на орбиту спутники и доставляется полезная нагрузка к Международной Космической Станции. Также в конструкторском бюро создаются аппараты научного и прикладного значения, позволяющие проводить научные исследования и эксперименты для решения фундаментальных и прикладных задач современной науки.\\
На основании вышеперечисленного представляется актуальным проведение постоянного мониторинга действий хозяйствующего субъекта.\\
Целью нашей работы является приобретение навыков анализа производственно-хозяйственной деятельности предприятия космической отрасли.\\
В качестве объекта практической работы возьмём акционерное общество \enquote{РКЦ \enquote{Прогресс}}.\\
Предметная область анализа --- это система управления и производственно-хозяйственная деятельность.\\
Цель работы достигается последовательным выполнением следующих задач:
\begin{enumerate}
    \item Изучение истории и хронологии развития организации.
    \item Исследование фактора внешней среды.
    \item чето там ещё\dots
    \item Изучение организационной структуры на предмет соответствия её целям и задачам.
\end{enumerate}
\pagebreak